\documentclass{article}
\usepackage{framed}
\usepackage[letterpaper, margin=1in]{geometry}
\usepackage{hyperref}

\setlength{\parindent}{0pt}
\hypersetup{colorlinks=true, linkcolor=blue, filecolor=magenta, urlcolor=cyan}

\title{The Kragle - Battlecode 25 Postmortem}
\author{Justin Ottesen, Andrew Bank, Matt Voynovich}
\date{Last Updated: \today}

\begin{document}

  \maketitle

  \section{Introduction}  

  \subsection{Our Team - The Kragle}

  We are a team of three computer science students at Rensselaer Polytechnic Institute (RPI). We are historically terrible at coming up with team names, and ended up choosing ``The Kragle'' after Justin had a burst of inspiration working on a \verb|MicroManager| class shortly after re-watching The Lego Movie. Just you wait for what we cook up next year. Some more information about us is below:
  
  \medskip

  \textbf{Andrew Bank} 

  \medskip

  \textbf{Justin Ottesen} has competed in Battlecode since 2022. He began his undergraduate degree in Computer Science in Fall 2021, and graduated early in Spring 2024. He stayed at RPI for his Master's, where he does research with the BRAINS Lab under Oshani Seneviratne with a focus on incentive design for smart contract protocols, and is on track to graduate in Fall 2025. Outside of classwork, he has worked as an intern at Nasuni since Summer 2023, and is on the D3 NCAA Cross Country and Track teams at RPI. In his free time, he enjoys hiking, running, programming, and playing Mario Kart Wii.

  \medskip

  \textbf{Matt Voynovich}

  \subsection{Past Performance}

  Andrew was the only member to compete in 2021. In 2022, Justin joined Andrew, working together again in 2023 and 2024 along with some other teammates. We did not perform particularly well in any of these years, typically holding a 1200-1500 rating, with no notable performances in any of the tournaments. Unfortunately, the RPI spring semester starts very early, so we were always balancing classwork along with the competition. This year we went all in, and entered the US qualifier tournament with the 10 seed, rated at 1730. I guess you'll have read this to see how we did...

  \medskip

  Although we aren't sure our mess of a repository will be useful to anyone else, our GitHub repositories can be found below:
  \begin{enumerate}
    \item[2021] \textbf{Fire Nation} - URL?
    \item[2022] \textbf{Kernel Byters} - \url{https://github.com/justinottesen/Kernel-Byters}
    \item[2023] \textbf{PC Ghosts} - \url{https://github.com/andrewkbank/bc2023}
    \item[2024] \textbf{Goat House} - \url{https://github.com/justinottesen/battlecode24}
    \item[2025] \textbf{The Kragle} - \url{https://github.com/justinottesen/battlecode25}
  \end{enumerate}

  \section{Battlecode 2025: Chromatic Conflict}

  \subsection{Game Overview}

  This year's introduction message is shown below:

  \begin{quote}
    \textit{The bread and food of yore has begun to run out, forcing robot society to adapt. Gone are the jovial ducks, replaced by steampunk robot bunnies who have converted their need for nutrients into a reliance on paint. These bunnies have become territorial, forming clans and defense formations to protect the resource that keeps them running.}
    
    \medskip
    
    \textit{For the past two centuries, these bunnies have stayed within their own territory, but clans have begun to degrade their environment and need to start branching out. Will these clans be able to expand their territory and generate enough paint to protect their families? Or will they stray too close to other clans and be wiped out in conflict?}
  \end{quote}

  As hinted above, this year's game was a competition of territory control between paint-crazy bunnies. The first team to paint 70\% of the map would win. Each team had two starting towers, which could produce robots. These robots had different abilities, but their goal was to work together to build more towers and paint as much territory as possible.

  \subsection{Map}

  The map style was rather simple this year. All maps were within $20 \times 20$ to $60 \times 60$ (inclusive) discrete tiles, and were guaranteed to be either vertically, horizontally, or rotationally symmetrical. Tiles were either empty, walls, or ruins. Robots could move on and paint empty tiles, but they could not for walls and ruins. As always in Battlecode, the map was 

  \subsection{Resources}

  The two resources of this year's game were Money (also referred to as Chips) and Paint.
  
  \medskip

  \textbf{Money} was a globally shared resource and could only be generated through towers. Money could be spent on new towers, tower upgrades, or new robots.
  
  \medskip

  \textbf{Paint} was a more complicated resource to manage, since it was held by individual units. Paint could be produced by towers or stolen from enemies, and could be spent to attack, expand territory, or produce new robots. When painted on the map, paint could be either a primary or secondary color for each team, and this was used to designate special patterns, which are described further below.

  \subsection{Towers}

  There were three tower types: Money towers, Paint towers, and Defense towers. Money and Paint towers passively produced their respective resources, and Defense towers initially just had high attack and health, but were buffed to also produce chips on every attack.

  \medskip

  Towers could do a weak attack to all enemies in their attack radius, and a stronger attack to a single enemy per turn. Towers could only be constructed at sites on the map called Ruins. Each tower type had its own $5 \times 5$ pattern that needed to be painted around a ruin before a tower could be built there. In addition, teams were limited to 25 towers at any given time.

  \subsection{Robots}

  There were three robot types: Soldiers, Splashers, and Moppers.

  \medskip

  \textbf{Soldiers} were the ``standard'' unit of this year. They could paint tiles to a team's color or attack enemy towers. Importantly, soldiers could not paint over enemy paint. They could only paint empty tiles, or allied tiles (which was useful in painting a pattern).

  \medskip

  \textbf{Splashers} could paint a group of tiles, regardless of their current paint status. However, they were more expensive than soldiers, and their attack was far less paint efficient. This attack also damaged enemy towers in range.

  \medskip

  \textbf{Moppers} could remove enemy paint from tiles. If an enemy was on the tile, it would also steal some of their paint and give it to the mopper. They also had the ability to swing their mop, which could attack close-by groups of clustered enemies, taking away enemy paint.

  \subsection{Special Resource Patterns}

  In addition to the tower patterns that could be painted on the map, robots could paint Special Resource Patterns (SRPs), which boosted the production rate of all friendly towers. A balance change caused these to have a 200 chip cost to complete, and would only activate after being completed for 50 consecutive turns. As soon as the pattern was disturbed, it would be deactivated.

  \subsection{Communication}

  Communication was relatively restricted this year. Robots could not communicate with each other, they could only communicate by messaging towers when they were close by and there was a path of friendly paint connecting them. Towers could send messages to robots and they could broadcast messaged to other towers within a slightly larger range. Robots could only send one message per round, and towers could send 20. The message consisted of the round number, the sending robot ID, and a 32 bit integer.

  \section{Strategy \& Implementation}

  

\end{document}