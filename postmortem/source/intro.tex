We believe our postmortem will be of unique help to future Battlecode participants. Often, teams who make postmortems are those who have consistently placed highly from the start. We have the perspective of a team that spent years with no consideration of qualifiers, who this year became one of the higher level contenders, going toe-to-toe with the 2 seed in the US qualifiers.

\medskip

Hopefully you find this useful!

\subsection{Our Team - The Kragle}

We are a team of three computer science students at Rensselaer Polytechnic Institute (RPI). In this year's competition, we made ``the leap'' from being a team that couldn't submit a final bot to being a contender for the final tournament. We'll outline some tips and tricks for other aspiring teams to make 'the leap' themselves. We are historically terrible at coming up with team names, and ended up choosing ``The Kragle'' after Justin had a burst of inspiration working on a \verb|MicroManager| class shortly after re-watching The Lego Movie. Just you wait for what we cook up next year. Some more information about us is below:

\medskip

\textbf{Andrew Bank} has competed in Battlecode since 2021. He is a Computer Science and Computer Systems Engineer at RPI and will graduate in this spring 2025. Andrew has interned at Johns Hopkins Applied Physics Lab, he has created the Circuit Randomizer for Personalized Learning as an undergraduate research project under Professor Shayla Sawyer, and is currently working on another undergraduate research project: the MusicX project for Professor Sawyer's Mercer Xlab. In his free time, Andrew enjoys having more Strava follwers than Instagram followers, and he enjoys treating Stardew Valley like an operating systems optimization problem while playing with his siblings.

\medskip

\textbf{Justin Ottesen} has competed in Battlecode since 2022. He began his undergraduate degree in Computer Science in Fall 2021, and graduated early in Spring 2024. He stayed at RPI for his Master's, where he does research with the BRAINS Lab under Oshani Seneviratne with a focus on incentive design for smart contract protocols, and is on track to graduate in Fall 2025. Outside of classwork, he has worked as an intern at Nasuni since Summer 2023, and is on the D3 NCAA Cross Country and Track teams at RPI. In his free time, he enjoys hiking, running, programming, and playing Mario Kart Wii.

\medskip

\textbf{Matthew Voynovich} started competing in Battlecode this year. He began his undergraduate degree majoring in both Computer Science and Information Technology and Web Science at RPI in Fall of 2023. He is scheduled to graduate in Spring 2026 and plans to pursue a masters under the Co-Terminal program at RPI. Currently, Matthew works as an undergrad researcher under Thomas Morgan studying quantum phase investigations, and in his free time enjoys long distance running with his friends in the Rensselaer Running Club.

\newpage
\subsection{Past Performance}

Andrew was the only member to compete in 2021. In 2022, Justin joined Andrew, working together again in 2023 and 2024 along with some other teammates. We did not perform particularly well in any of these years, typically holding a 1200-1500 rating, with no notable performances in any of the tournaments. Unfortunately, the RPI Spring semester always starts very early, so we were always balancing classwork along with the competition. This year we went all in, and entered the US qualifier tournament with the 10 seed, rated at 1730. I guess you'll have read this to see how we did...

\medskip

Although we aren't sure our code will be useful to anyone else, our GitHub repositories can be found below:
\begin{enumerate}
  \item[2021] \textbf{Fire Nation} - Lost to the sands of time\dots
  \item[2022] \textbf{Kernel Byters} - \url{https://github.com/justinottesen/Kernel-Byters}
  \item[2023] \textbf{PC Ghosts} - \url{https://github.com/andrewkbank/bc2023} (This is still private)
  \item[2024] \textbf{Goat House} - \url{https://github.com/justinottesen/battlecode24}
  \item[2025] \textbf{The Kragle} - \url{https://github.com/justinottesen/battlecode25} (This is still private)
\end{enumerate}

\subsection{Game Overview}

This year's introduction message is shown below:

\begin{quote}
  \textit{The bread and food of yore has begun to run out, forcing robot society to adapt. Gone are the jovial ducks, replaced by steampunk robot bunnies who have converted their need for nutrients into a reliance on paint. These bunnies have become territorial, forming clans and defense formations to protect the resource that keeps them running.}
  
  \medskip
  
  \textit{For the past two centuries, these bunnies have stayed within their own territory, but clans have begun to degrade their environment and need to start branching out. Will these clans be able to expand their territory and generate enough paint to protect their families? Or will they stray too close to other clans and be wiped out in conflict?}
\end{quote}

As hinted above, this year's game was a competition of territory control between paint-crazy bunnies. The first team to paint 70\% of the map would win. Each team had two starting towers, which could produce resources and robots. These robots had different abilities, but their goal was to work together to build more towers and paint as much territory as possible.

\medskip

We saved copies of both the initial and final specs to our repository in case they are taken down in the future. For a full description of the game, see \href{https://github.com/justinottesen/battlecode25}{our repository}. If you are new to Battlecode, we highly recommend \href{https://battlecode.org/assets/files/battlecode-guide-xsquare.pdf}{this Battlecode Guide}, written by XSquare, a long time Battlecode competitor.